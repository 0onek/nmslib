\pdfoutput=1
\documentclass[runningheads,a4paper]{llncs}

\usepackage{amssymb}
\setcounter{tocdepth}{3}
\usepackage{graphicx}
\usepackage{amsmath}
\usepackage{hyperref}
\usepackage{todonotes}
\usepackage{listings}
\usepackage{xcolor}
\usepackage{wrapfig}
\usepackage{subfig}
\usepackage{booktabs}
\usepackage{url}


\begin{document}

\mainmatter  % start of an individual contribution

% first the title is needed
\title{Non-Metric Space Library}

% a short form should be given in case it is too long for the running head
\titlerunning{Non-Metric Space Library Manual}

% the name(s) of the author(s) follow(s) next
%
% NB: Chinese authors should write their first names(s) in front of
% their surnames. This ensures that the names appear correctly in
% the running heads and the author index.
%
\author{Leonid Boytsov\inst{1} \and Bilegsaikhan Naidan\inst{2}}
%
\authorrunning{Bilegsaikhan Naidan and Leonid Boytsov}
% (feature abused for this document to repeat the title also on left hand pages)

% the affiliations are given next; don't give your e-mail address
% unless you accept that it will be published
\institute{
Department of Computer and Information Science,\\
Norwegian University of Science and Technology,\\
Trondheim, Norway\\
\email{bileg@idi.ntnu.no}\\
{\hspace{1em}}\\
\and
Language Technologies Institute, \\
Carnegie Mellon University,\\
Pittsburgh, PA, USA\\
\email{leo@boytsov.info}\\
}

%
% NB: a more complex sample for affiliations and the mapping to the
% corresponding authors can be found in the file "llncs.dem"
% (search for the string "\mainmatter" where a contribution starts).
% "llncs.dem" accompanies the document class "llncs.cls".
%

%\toctitle{Lecture Notes in Computer Science}
%\tocauthor{Authors' Instructions}

\maketitle

{\begin{center}{December 2013}\end{center}}

\begin{abstract}
This document describes a library for similarity searching.
Even though it contains a variety of metric-space access methods,
our main focus is searching in non-metric spaces.
Because there are fewer exact solutions for non-metric spaces,
many of our methods give only approximate answers and 
are evaluated in terms of efficiency-effectiveness trade-offs
rather than merely in terms of their run-time.
We concentrate on technical details, i.e., 
how to compile the code, run the benchmarks, and evaluate results.
Additionally, we explain how to extend the code by adding
new search methods and spaces.
\end{abstract}

\bibliographystyle{abbrv}
%\bibliographystyle{plain}
%\bibliography{manual}

\section{Introduction}

\end{document}
